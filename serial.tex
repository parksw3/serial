\documentclass[12pt]{article}
\usepackage[top=1in,left=1in, right = 1in, footskip=1in]{geometry}

\usepackage{graphicx}
%\usepackage{adjustbox}

%% \newcommand{\comment}{\showcomment}
\newcommand{\comment}{\nocomment}

\newcommand{\showcomment}[3]{\textcolor{#1}{\textbf{[#2: }\textsl{#3}\textbf{]}}}
\newcommand{\nocomment}[3]{}

\newcommand{\jd}[1]{\comment{cyan}{JD}{#1}}
\newcommand{\swp}[1]{\comment{magenta}{SWP}{#1}}

\newcommand{\eref}[1]{Eq.~\ref{eq:#1}}
\newcommand{\fref}[1]{Fig.~\ref{fig:#1}}
\newcommand{\Fref}[1]{Fig.~\ref{fig:#1}}
\newcommand{\sref}[1]{Sec.~\ref{#1}}
\newcommand{\frange}[2]{Fig.~\ref{fig:#1}--\ref{fig:#2}}
\newcommand{\tref}[1]{Table~\ref{tab:#1}}
\newcommand{\tlab}[1]{\label{tab:#1}}
\newcommand{\seminar}{SE\mbox{$^m$}I\mbox{$^n$}R}

\usepackage{amsthm}
\usepackage{amsmath}
\usepackage{amssymb}
\usepackage{amsfonts}

% \usepackage{lineno}
% \linenumbers

\usepackage[pdfencoding=auto, psdextra]{hyperref}

\usepackage{natbib}
\bibliographystyle{chicago}
\date{\today}

\usepackage{xspace}
\newcommand*{\ie}{i.e.\@\xspace}

\usepackage{color}

\newcommand{\Rx}[1]{\ensuremath{{\mathcal R}_{#1}}} 
\newcommand{\Ro}{\Rx{0}}
\newcommand{\RR}{\ensuremath{{\mathcal R}}}
\newcommand{\Rhat}{\ensuremath{{\hat\RR}}}
\newcommand{\tsub}[2]{#1_{{\textrm{\tiny #2}}}}

\begin{document}

\begin{flushleft}{
	\Large
	\textbf\newline{
		Understanding serial interval distributions: applications to the COVID-19 pandemic
	}
}
Sang Woo Park
\end{flushleft}

\section*{Abstract}

\pagebreak

%The key distributions for understanding the spread of COVID-19 include:
%\begin{enumerate}
%  \item Incubation period distribution: time between infection and symptom onset \citep{backer2020incubation, %li2020early, linton2020incubation, tian2020characteristics}
%  \item Serial interval distribution: time between symptom onset of an infector and an infectee \citep{du2020s%erial, nishiura2020serial, zhao2020estimating}
%  \item Generation interval distribution: time between infection of an infector and an infectee \citep{ganyani%2020estimating}
%\end{enumerate}

\section{Introduction}

Since the emergence of the novel coronavirus disease (COVID-19), a significant amount of research has focused on estimating its basic reproduction number $\mathcal R_0$.
The basic reproduction number, which is defined as the average number of secondary cases caused by a primary case, allows us to predict the extent to which a disease will spread and the amount of intervention to prevent an outbreak.
Since the basic reproduction number cannot be measured directly, it is often inferred from the observed exponential growth rate using generation or serial intervals.

Generation intervals are defined as the time between when an individual is infected and when an individual infects another person.
Similarly, serial intervals are defined as the time between when an infector and an infectee become symptomatic.
While 

We show that the serial-interval distribution can provide a correct link between the speed 


\section{Backward and forward delay distributions}


\bibliography{serial}

\end{document}
