\documentclass[12pt]{article}
\usepackage[top=1in,left=1in, right = 1in, footskip=1in]{geometry}

\usepackage{graphicx}
%\usepackage{adjustbox}

%% \newcommand{\comment}{\showcomment}
\newcommand{\comment}{\nocomment}

\newcommand{\showcomment}[3]{\textcolor{#1}{\textbf{[#2: }\textsl{#3}\textbf{]}}}
\newcommand{\nocomment}[3]{}

\newcommand{\jd}[1]{\comment{cyan}{JD}{#1}}
\newcommand{\swp}[1]{\comment{magenta}{SWP}{#1}}

\newcommand{\eref}[1]{Eq.~\ref{eq:#1}}
\newcommand{\fref}[1]{Fig.~\ref{fig:#1}}
\newcommand{\Fref}[1]{Fig.~\ref{fig:#1}}
\newcommand{\sref}[1]{Sec.~\ref{#1}}
\newcommand{\frange}[2]{Fig.~\ref{fig:#1}--\ref{fig:#2}}
\newcommand{\tref}[1]{Table~\ref{tab:#1}}
\newcommand{\tlab}[1]{\label{tab:#1}}
\newcommand{\seminar}{SE\mbox{$^m$}I\mbox{$^n$}R}

\usepackage{amsthm}
\usepackage{amsmath}
\usepackage{amssymb}
\usepackage{amsfonts}

% \usepackage{lineno}
% \linenumbers

\usepackage[pdfencoding=auto, psdextra]{hyperref}

\usepackage{natbib}
\bibliographystyle{chicago}
\date{\today}

\usepackage{xspace}
\newcommand*{\ie}{i.e.\@\xspace}

\usepackage{color}

\newcommand{\Rx}[1]{\ensuremath{{\mathcal R}_{#1}}} 
\newcommand{\Ro}{\Rx{0}}
\newcommand{\RR}{\ensuremath{{\mathcal R}}}
\newcommand{\Rhat}{\ensuremath{{\hat\RR}}}
\newcommand{\tsub}[2]{#1_{{\textrm{\tiny #2}}}}

\begin{document}

\begin{flushleft}{
	\Large
	\textbf\newline{
		Unraveling the paradox between generation and serial intervals: applications to COVID-19 pandemic
	}
}\\
Sang Woo Park
\end{flushleft}

\section*{Abstract}

\pagebreak

%The key distributions for understanding the spread of COVID-19 include:
%\begin{enumerate}
%  \item Incubation period distribution: time between infection and symptom onset \citep{backer2020incubation, %li2020early, linton2020incubation, tian2020characteristics}
%  \item Serial interval distribution: time between symptom onset of an infector and an infectee \citep{du2020s%erial, nishiura2020serial, zhao2020estimating}
%  \item Generation interval distribution: time between infection of an infector and an infectee \citep{ganyani%2020estimating}
%\end{enumerate}

\section{Introduction}

Since the emergence of the novel coronavirus disease (COVID-19), a significant amount of research has focused on estimating its reproduction number $\mathcal R$.
The reproduction number, which is defined as the average number of secondary cases caused by a primary case, allows us to predict the extent to which a disease will spread in the population and the amount of intervention to prevent an outbreak.
Since the reproduction number cannot be measured directly, it is often estimated from the observed exponential growth rate using generation- and serial-interval distributions.

The generation interval is defined as the time between when an individual (infector) is infected and when an individual infects another person (infectee).
On the other hand, the serial interval is defined as the time between when an infector and an infectee become symptomatic.
% Despite having similar definitions, generation and serial intervals are different because serial intervals rely on the generation intervals as well as the incubation periods of an infector and an infectee.
Previous studies have often alluded to differences in their variances, despite having identical expected values, and noted that using serial-interval distribution can give different estimates of $\mathcal R$.
Even though these distinctions were first made over a decade ago \citep{svensson2007note}, 
the need for a better conceptual and theoretical framework for understanding their differences is becoming clearer as the COVID-19 pandemic unfolds:
Researchers continue to rely on both generation and serial intervals to estimate $\mathcal R$ for COVID-19 without making a clear distinction, and some even mislabel them.

The lack of clear understanding of their differences can be largely attributed to an apparent paradox.
When the epidemic is growing exponentially, the spread of infection can be characterized as a ``renewal process'' based on previous incidence of infection, the associated generation-interval distribution, and the average infectiousness of an infected individual.
This renewal formulation allows us to link the exponential growth rate of an epidemic $r$ with its reproduction number $\mathcal R$.
Likewise, since infection typically leads to symptoms, we should be able to describe the renewal process of symptomatic cases using the serial-interval distribution.
Therefore, both generation- and serial-interval distributions should give us identical estimates of  $\mathcal R$ based the observed epidemic growth rate.
Current theory seems to contradict our biological intution.

Here, we provide an answer to the decade-old paradox.
We provide a new framework for understanding serial intervals and show that both serial- and generation-interval distributions give identical estimates of $\mathcal R$ during the exponential growth phase of an epidemic.

\section{Methods}

\subsection{Backward and forward delay distributions}

We first begin by describing a general framework for characterizing a distribution of time delays between \emph{any} two epidemiological events;
these events can be defined either within an infected individual (e.g., infection and symptom onset of an individual) or between infected individuals (e.g., symptom onsets of an infector and an infectee).
Then, we can further divide these events into \emph{primary} and \emph{secondary} events.
When we measure an epidemiological time delay within an infected individual (e.g., time between infection and symptom onset), the primary event always occurs before the secondary event.
When we measure an epidemiological time delay between infected individuals (e.g., serial interval), 
the primary and secondary events are defined in terms of the direction of transmission:
The primary event refers to the event that occurs within an infector and does not necessarily occur before the secondary event.

We model time delays between a primary and a secondary event from a cohort perspective.
A primary cohort consists of \emph{all} individuals whose primary event occurred at a given time; 
a secondary cohort can be defined similarly based on the secondary events.
For example, when we are measurign serial intervals, a primary cohort $s$ consists of all infectors who became symptomatic at time $s$.
Then, for each primary cohort $s$, we can define the expected time distribution between primary and secondary events.
We refer to this distribution as the forward delay distribution and denote it as $f_s(\tau)$.
We assume that forward delay distributions can vary across primary cohorts.

Likewise, we can define a backward delay distribution $b_s(\tau)$ for a secondary cohort $s$:
The backward delay distribution for secondary host $s$ describes the time delays between a primary and secondary host given that the secondary event occurred at time $s$.
It follows that:
\begin{equation}
b_s(\tau) \propto C(s-\tau) f_{s-\tau}(\tau)
\end{equation}
where $C(s)$ is the size of the primary cohort $s$.
Therefore, changes in the backward delay distribution depends on the changes in cohort size $C(s)$ (therefore incidence of infection) as well as changes in the forward delay distribution.
These ideas apply to all epidemiological delay distributions and generalize the work by \citep{champredon2015intrinsic} who compared forward and backward generation-interval distributions to describe the realized generation intervals from the perspective of an infector and an infectee, respectively.

\subsection{Realized serial interval distributions}

The serial interval is defined as the time between when an infector becomes symptomatic and when and infectee becomes symptomatic.
Here, we denote realized serial intervals $\tau$ as:
\begin{equation}
\tau = - x_0 + \sigma + x_1
\end{equation}
where $x_0$ and $x_1$ represent the realized time from infection to symptom onset of an infector and an infectee, respectively, and $\sigma$ represents the realized generation interval.
Previous studies have assumed that (i) $x_0$ and $x_1$ have the same means and (ii) therefore the serial and generation intervals have the same mean;
however, these results implicitly assume that incidence stays constant.

Using the cohort-based framework provides a clear way of understanding the serial-interval distribution.
Given that an infector became symptomatic at time $\ell$, we have to first go backward in time by asking when the infector was infected and go forward in time by asking when the infectee became symptomatic.
Then, it is clear that $x_0$ follows the backward incubation period distribution for secondary cohort $\ell$; 
$\sigma$ follows the forward generation-interval distribution for a primary cohort $\ell - x_0$ conditional on the incubation period  $x_0$ of the infector;
and $x_1$ follows forward incubation period distribution for a primary cohort $\ell - x_0 + \sigma$.
Assuming that the forward incubation distribution does not vary across cohorts, the forward serial interval distribution of a primary cohort $\ell$ can be written as follows:
\begin{equation}
f_\ell(\tau) \propto \int_{0}^\infty \int_{0}^\infty i(\ell - x_0) h_{\ell - x_0}(x_0, \sigma) k(\tau-\sigma+x_0) \mathrm{d} x_0\, \mathrm{d}\sigma
\end{equation}
where $i$ is the incidence of infection, $h$ is the joint probability distribution of the incubation period and generation interval of a primary cohort $\ell - x_0$, and $k$ is a marginal probability distribution of $h$ describing the forward incubation periods.
Therefore, the forward serial interval distribution depends on previous changes in incidence and generally have different means from the generation interval distribution.

Likewise, we can define backward serial interval distribution for a secondary cohort $\ell$.
Given that an infectee became symptomatic at time $\ell$, we have to first go backward in time by asking when the infectee became infected and when the infector became infected; 
then, we have to go forward in time by asking when the infector became symptomatic.
In this case, $x_1$ and $\sigma$ follow the backward incubation period and generation-interval distributions, respectively, and $x_0$ follows the forward incubation period distribution.
Therefore, the backward serial interval distribution of a secondary cohort $\ell$ can be written as follows
\begin{equation}
\begin{aligned}
b_\ell(\tau) &\propto j(\ell) f_{\ell}(\tau)\\
&\propto \int_0^\infty i(\ell-x_0) k(x_0) f_{\ell}(\tau) \mathrm{d} x_0
\end{aligned}
\end{equation}
where $j$ represents the incidence of symptomatic cases.

\subsection{Observed serial interval distributions}



\subsection{Relating $\mathcal R$ and $r$ using generation and serial-interval distributions}

Assuming that the forward generation-interval distribution remains constant during the exponential growth phase,
we can estimate the reproduction number from the exponential growth rate $r$ via the Euler-Lotka equation:
\begin{equation}
\frac{1}{\mathcal R} = \int_0^\infty \exp(-r\tau) g(\tau) \mathrm{d} \tau,
\end{equation}
where $g(\tau)$ is the forward generation-interval distribution during the exponential growth phase (and also a marginal probability distribution of $h$).
Like the forward generation-interval distribution, 
the forward serial-interval distribution describes the renewal process of symptomatic cases.
Therefore, the forward serial-interval distribution $f_{\tiny\textrm{exp}}(\tau)$ during the exponential growth phase should provide the identical $r$--$\mathcal R$ link as the generation-interval distribution:
\begin{equation}
\frac{1}{\mathcal R} = \int_{-\infty}^\infty \exp(-r\tau) f_{\tiny\textrm{exp}}(\tau) \mathrm{d} \tau,
\end{equation}
where the forward serial-interval distribution during the exponential growth phase is defined as:
\begin{equation}
f_{\tiny\textrm{exp}}(\tau) \propto \int_{0}^\infty \int_{0}^\infty \exp( -r x_0) h(x_0, \sigma) k(\tau-\sigma+x_0) \mathrm{d} x_0\, \mathrm{d}\sigma.
\end{equation}
In Appendix, we provide a mathematical proof that this relationship holds.

We note that the forward serial-interval distribution depends on the exponential growth rate $r$.
When the epidemic grows fast (high $r$), we expect the backward incubation period to be short, and therefore, the forward serial-interval distribution will generally have a longer mean than the forward generation-interval distribution.
The Susceptible-Exposed-Infected-Recovered model, which assumes that incubation and exposed periods are equivalent, is a special case where the conditional forward generation-interval distribution cancels out with the backward generation-interval distribution exactly because (i) infected individuals can only transmit after symptom onset and (ii) the incubation period distribution is independent of the time between symptom onset to infection;
in this case, the forward serial- and generation-interval distributions follow identical distributions.



\section{Results}

\section{Appendix}

Recall that the forward serial interval can be written as:
\begin{equation}
- x_0 + \sigma + x_1
\end{equation}
Note that $x_1$ is independent of $x_0$ and $\sigma$. Then, we get:
\begin{equation}
M_{- x_0 + \sigma + x_1}(-r) = M_{- x_0 + \sigma}(-r) M_{x_1}(-r).
\end{equation}
We want to show that 
\begin{equation}
M_{- x_0 + \sigma}(-r)= \frac{M_\sigma(-r)}{M_{x_1}(-r)}
\end{equation}
for $r \geq 0$.
Note that the time between symptom onset and infection of an infectee follows the following distribution:
\begin{equation}
z(\tau) \propto \int_{\max(0, -\tau)}^\infty i(\ell - x_0) h_{\ell - x_0}(x_0, \tau+x_0) \mathrm{d} x_0.
\end{equation}
During the exponential growth phase, we have
\begin{equation}
\begin{aligned}
z_{\tiny\textrm{exp}}(\tau) &= \frac{1}{N} \int_{\max(0, -\tau)}^\infty \exp(- r x_0) h(x_0, \tau+x_0) \mathrm{d} x_0\\
% &= \frac{1}{N} \int_{0}^\infty \left[ \exp(- r x_0) k(x_0) \times \frac{h(x_0, \tau+x_0)}{k(x_0)}\right] \mathrm{d} x_0
\end{aligned}
\end{equation}
where $N$ is the normalization factor:
\begin{equation}
\begin{aligned}
N &= \int_{-\infty}^\infty \int_{\max(0, -\tau)}^\infty \exp(- r x_0) h(x_0, \tau+x_0) \mathrm{d} x_0\,\mathrm{d}\tau\\
&= \int_{0}^\infty \int_{-x_0}^\infty \exp(- r x_0) h(x_0, \tau+x_0) \mathrm{d}\tau\,\mathrm{d} x_0\\
&= \int_{0}^\infty \exp(- r x_0) k(x_0) dx_0\\
&= M_{x_1}(-r)
\end{aligned}
\end{equation}
Therefore,
\begin{equation}
\begin{aligned}
&\int_{-\infty}^{\infty} \exp(-r\tau) z_{\tiny\textrm{exp}}(\tau) \mathrm{d}\tau\\
&=\frac{1}{M_{-x_1}(-r)} \int_{-\infty}^{\infty} \exp(-r\tau) \int_{\max(0, -\tau)}^\infty \exp(- r x_0) h(x_0, \tau+x_0) \mathrm{d} x_0\, \mathrm{d}\tau
\end{aligned}
\end{equation}

Finally, we are left to prove that 
\begin{equation}
\int_0^{\infty} \exp(-r\tau) g(\tau) \mathrm{d}\tau = \int_{-\infty}^{\infty} \exp(-r\tau) \int_{\max(0, -\tau)}^\infty \exp(- r x_0) h(x_0, \tau+x_0) \mathrm{d} x_0\, \mathrm{d}\tau
\end{equation}
where $g$ is a marginal probability distribution of $h$ describing the forward generation intervals:
\begin{equation}
g(\tau) = \int_0^\infty h(x_0, \tau)  \mathrm{d} x_0.
\end{equation}
Let $a = x_0 + \tau$. Then, by change of variables, it immediately follows that
\begin{equation}
\begin{aligned}
&\int_{-\infty}^{\infty} \exp(-r\tau) \int_{\max(0, -\tau)}^\infty \exp(- r x_0) h(x_0, \tau+x_0) \mathrm{d} x_0\, \mathrm{d}\tau\\
&=\int_{0}^{\infty} \int_{0}^\infty \exp(- r a) h(x_0, a) \mathrm{d} x_0\, \mathrm{d}a\\
&=\int_{0}^{\infty} \exp(-r\tau) g(\tau) \mathrm{d}\tau
\end{aligned}
\end{equation}
Therefore, the forward serial- and generation-interval distributions give the same link between $r$ and $\mathcal R$.

\bibliography{serial}

\end{document}
